%! Author = air
%! Date = 10.10.2022

{}\documentclass{article}
\usepackage[T2A]{fontenc}
\usepackage[utf8]{inputenc}
\usepackage[russian]{babel}
\usepackage{tikz}
\usepackage{amscd}
\usepackage[inline]{enumitem}
\usepackage{amsmath}
\usepackage{dsfont}
\usepackage{indentfirst}
\usepackage{amssymb}
\usepackage{amsfonts}
\usepackage{amsthm}
\usepackage{epigraph}
\usepackage{icomma}
\renewcommand{\thesection}{\arabic{section}}
\renewcommand{\baselinestretch}{1.0}
\renewcommand\normalsize{\sloppypar}
\setlength{\topmargin}{-0.5in}
\setlength{\textheight}{9.1in}
\setlength{\oddsidemargin}{-0.3in}
\setlength{\textwidth}{7in}
\setlength{\parindent}{0ex}
\setlength{\parskip}{1ex}
\newtheorem{theorem}{Теорема}
\newtheorem{utv}{Утверждение}
\newtheorem{lemma}{Лемма}
% Document
\begin{document}
    \textbf{Основные формулы}

    \textbf{Математическое ожидание} дискретной случайной величины:

    \[
        \mathbb{E}\left(X\right) = \sum_{i=1}^{n} x_i \cdot p_i
    \]

    \textbf{Дисперсия} дискретной случайной величины:

    \[
        \mathbb{D}\left(X\right) = \mathbb{E} \left( X - \mathbb{E}\left( X \right)^2 \right)
    \]

    \textbf{Среднее квадратическое отклонение}:

    \[
        \sigma = \sqrt {\frac{1}{n} \sum_{i=1}^{n} \left( x_i - \overline{x} \right)^2}
    \]

    \textbf{Неравенство Маркова}:

    \[
        P_{\left\{ \xi \left\{ \omega \right\} \geq a \right\}} \leq \frac{\mathbb{E} f\left( \xi \right)}{b}
    \]

    \textbf{Неравенство Чебышева}:

    \[
        P_{\left\{ \xi \left\{ \omega \right\} \geq a \right\}} \leq \frac{\mathbb{E} f\left( \xi^2 \right)}{a^2}
    \]

    \textbf{Задание 1:}
    Среднее количество вызовов, поступающих на коммутатор завода в течение часа, равно 300.
    Оценить вероятность того, что в течение следующего часа число вызовов на коммутатор:
    а) превысит 400;
    б) будет не более 500.

    \textbf{Дано:}
    $\mathbb{E}x = 300$

    \textbf{Найти:}
    a) $P(x > 400)$
    б) $P(x \leq 500)$

    \textbf{Решение:}

    Применим Неравенство Маркова

    а) $P(x \geq 400) \leq \frac{300}{400} = 0.75 \Rightarrow$

    $P(x > 400) \leq 0.75 - P(x = 400) < 0.75$

    \textbf{Ответ:} $P(x > 400) < 0.75$ \\

    б) $P(x \leq 500) = 1 - P(x > 500)$

    $P(x \geq 500) \leq \frac{300}{500} = 0.6$

    $P(x > 500) \leq 0.6 - P(x = 500) < 0.6$

    $P(x \leq 500) \geq 1 - 0.6 = 0.4$

    \textbf{Ответ:} $P(x \leq 500) \geq 0.4$

\newpage

    \textbf{Задание 2:}
    В 1600 испытаниях Бернулли вероятность успеха в каждом испытании равна 0,3.
    С помощью неравенства Чебышёва оценить вероятность того, что разница между
    числом успехов в этих испытаниях и средним числом успехов будет меньше 50

    \textbf{Дано:}

    $n = 1600$

    $\xi \sim Bernoulli(p)$

    $p = 0.3$

    $\varepsilon = 50$

    \textbf{Найти:}

    $P(|\xi - \mathbb E {\xi}| < \varepsilon)$

    \textbf{Решение:}

    $\mathbb{E} \xi  = np = 1600 \cdot 0.3 = 480$

    $\mathbb{D} \xi = np(1-p) = 336$

    $P(|\xi - \mathbb{E}\xi| < \varepsilon) \geq 1 - \frac{\mathbb{D}\xi}{\varepsilon^2}$ - неравенство Чебышева

    $P(|\xi - 480| < 50) \geq 1 - \frac{336}{50^2} = 0.866$

    \textbf{Ответ:} $P(|\xi - \mathbb{E}\xi| < \varepsilon) = 0.866$

\newpage

    \textbf{Задание 3:}
    Дана выборка $9, 5, 7, 7, 4, 10$.
    Дисперсия $\mathbb{D} = 1$.
    Постройте $99 \%$ доверительный интервал

    \textbf{Дано:}

    $X \in \{9, 5, 7, 7, 4, 10\}$

    $\mathbb{D}X = 1$

    \textbf{Найти:}

    $99 \%$ доверительный интервал

    \textbf{Решение:}

    $\overline{X} = \frac{\sum_{i=1}^6 x_i}{6} = \frac{9+5+7+7+4+10}{6} = \frac{42}{6} = 7$

    $ 1 - \frac{\alpha}{2} = 1 - 0.005 = 0.995$

    $z_{0.995} = 2.58$ -квантиль

    $\Delta = \frac{\mathbb{D}X}{\sqrt{k}}z_{1 - \frac{\alpha}{2}} = \frac{2.58}{\sqrt{6}} = 1.053$

    $\overline{X} - \Delta = 5.947$

    $\overline{X} + \Delta = 8.053$

    $99 \%$ доверительный интервал: $(\overline{X} - \Delta, \overline{X} + \Delta) = (5.947; 8.053)$

    \textbf{Ответ:} $(5.947; 8.053)$

\newpage

    \textbf{Задание 4:}
    Пусть $X_i$ подчиняется нормальному распределению $N(\mu, \sigma^2)$.
    Найти ОМП $\hat{\mu}$ и $\hat{\sigma}$.

    Почему найденное решение будет точкой максимума функции прадоподобия, а не
    седловой точкой?

    \textbf{Дано:}

    $X_i \sim N(\mu, \sigma^2)$

    \textbf{Найти:}

    ОМП $\hat{\mu}$

    ОМП $\hat{\sigma}$

    \textbf{Доказать:}

    $(\hat{\mu}, \hat{\sigma})$ -- точка максимума функции правдоподобия

    \textbf{Решение:}

    $f_X(x; \mu, \sigma^2) = \frac{1}{\sqrt{2 \pi \sigma^2}} \exp{\bigg\left( \frac{-(x-\mu)^2}{2\sigma^2} \bigg\right)}$

    \begin{gather*}
        f_X(x; \mu, \sigma^2) =
    \Pi_{i=1}^n \frac{1}{\sqrt{2 \pi \sigma^2}} \exp{\bigg( \frac{-(x_i-\mu)^2}{2\sigma^2} \bigg)} =
    \bigg(\frac{1}{2 \pi \sigma^2}\bigg)^{\frac{n}{2}} \exp{\bigg( \frac{-\sum_{i=1}^n(x_i-\mu)^2}{2\sigma^2} \bigg)}\\
        \ln(L(X)) = -\frac{n}{2} \ln(2\pi) - \frac{n}{2} \ln (\sigma^2) - \sum_{i=1}^n \bigg(\frac{(x_i-\mu)^2}{2\sigma^2}
    \bigg)\\
        \frac{\partial \ln(L(X))}{\partial \mu} = -\frac{1}{2\sigma^2} \sum_{i=1}^n 2(x_i - \mu)(-1) = \sum_{i=1}^n \frac{x_i - \mu}{\sigma^2} = \frac{\sum_{i=1}^n x_i - n\mu}{\sigma^2} = \frac{n\overline{X} - n\mu}{\sigma^2} \Rightarrow \hat{\mu} = \overline{X}\\
        \frac{\partial \ln(L(X))}{\partial \sigma} = -n \cdot \frac{1}{\sigma} + \frac{1}{\sigma^3} \sum_{i=1}^n (x_i - \mu)^2 = \frac{\sum_{i=1}^n(x_i - \mu)^2 - n\sigma^2}{\sigma^3} = 0
    \Rightarrow
    \hat{\sigma}^2 = \frac{\sum_{i=1}^n (x_i - \hat{\mu})^2}{n}
    \Rightarrow
    \hat{\sigma} = \sqrt{\frac{\sum_{i=1}^n(x_i - \hat{\mu})^2}{n}}\\
    \end{gather*}

    Для того, чтобы найденная точка $(\hat{\mu}, \hat{\sigma})$ была точкой максимума, достаточно, чтобы все её вторые частные производные были отрицательны.

    \begin{gather*}
        \frac{\partial^2 \ln(L(X))}{\partial \mu^2} = -\frac{n}{\sigma^2} < 0 \ \forall (\mu, \sigma)\\
        \frac{\partial^2 \ln(L(X))}{\partial \sigma^2} = \frac{n}{\sigma^2} - \frac{3}{\sigma^4} \sum_{i=1}^n \bigg( x_i - \mu \bigg)^2 =
    \frac{n\sigma^2 - 3\sum_{i=1}^n (x_i - \mu)^2 }{\sigma^4} \bigg |_{\sigma = \hat{\sigma}} =
    \frac{\sum_{i=1}^n (x_i - \mu)^2 - 3\sum_{i=1}^n (x_i - \mu)^2 }{\sigma^4} =\\
        = -\frac{2\sum_{i=1}^n (x_i - \mu)^2}{\sigma^4} < 0\\
    \end{gather*}

Следовательно, поскольку обе производные отрицательны для любой точки, то найденная точка будет точкой максимума, а не седловой точкой.



\textbf{Ответ:} $\hat{\mu} = \overline{X}, \hat{\sigma} = \sqrt{\frac{\sum_{i=1}^n(x_i - \hat{\mu})^2}{n}}$

\end{document}
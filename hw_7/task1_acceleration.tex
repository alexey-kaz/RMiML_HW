%! Author = air
%! Date = 10.10.2022

{}\documentclass{article}
\usepackage[T2A]{fontenc}
\usepackage[utf8]{inputenc}
\usepackage[russian]{babel}
\usepackage{tikz}
\usepackage{amscd}
\usepackage[inline]{enumitem}
\usepackage{amsmath}
\usepackage{dsfont}
\usepackage{indentfirst}
\usepackage{amssymb}
\usepackage{amsfonts}
\usepackage{amsthm}
\usepackage{epigraph}
\usepackage{icomma}
\renewcommand{\thesection}{\arabic{section}}
\renewcommand{\baselinestretch}{1.0}
\renewcommand\normalsize{\sloppypar}
\setlength{\topmargin}{-0.5in}
\setlength{\textheight}{9.1in}
\setlength{\oddsidemargin}{-0.3in}
\setlength{\textwidth}{7in}
\setlength{\parindent}{0ex}
\setlength{\parskip}{1ex}
\newtheorem{theorem}{Теорема}
\newtheorem{utv}{Утверждение}
\newtheorem{lemma}{Лемма}
% Document
\begin{document}
    Обновление стохастического градиентного спуска (\textbf{SGD}) на каждой итерации $t$:

    \[
        \theta_{t+1} = \theta_t - \eta g
    \]

    где:

    \begin{itemize}
        \item $\theta$ -- параметр, который алгоритм будет изменять для достижения приемлемых потерь;
        \item $g$ -- градиент, который показывает противоположное направление
              и насколько сильно требуется менять параметры, чтобы минимизировать потери;
        \item $\eta$ -- скорость обучения – то, насколько мы изменяем наши параметры по отношению к градиенту.
    \end{itemize}

    \textbf{SGD} может быть медленным, например, когда градиент постоянно мал.
    Это связано с правилом обновления алгоритма, которое на каждой итерации зависит только от градиентов.
    `Шум` при подсчете градиента также может быть проблемой, поскольку стохастический градиентный спуск
    часто будет следовать неправильному градиенту.

    При обучении нейронных сетей может быть использован градиентный спуск на основе импульса (\textbf{Momentum})
    для борьбы с этими проблемами и ускорения обучения по сравнению с оригинальным \textbf{SGD}\@.
    Импульс учитывает предыдущие градиенты в правиле обновления на каждой итерации.

    \begin{align}
        & v_{t+1} = \alpha v_t - \eta g \\
        & \theta_{t+1} = \theta_t + v_{t+1}
    \end{align}

    где:

    \begin{itemize}
        \item $v$ -- направление и скорость, с которой параметр должен быть изменен
        \item $\alpha$ -- гиперпараметр затухания, который определяет, как быстро будут затухать накопленные градиенты \\
        \end{itemize}
    Если $\alpha$ намного больше $\eta$, накопленные градиенты будут доминировать в правиле обновления,
    поэтому градиент не изменит направление слишком быстро.
    Это хорошо в условиях, когда градиент зашумлен, потому что градиент навсегда останется в истинном направлении.
    С другой стороны, если $\alpha$ намного меньше $\eta$, накопленные градиенты могут действовать как
    фактор сглаживания градиента.

    Другим методом, тесно связанным с градиентным спуском на основе импульса,
    является ускоренный градиентный спуск Нестерова (\textbf{NAG}).
    Разница между \textbf{Momentum} и \textbf{NAG} заключается в фазе вычисления градиента.
    В методе импульса градиент был вычислен с использованием текущих параметров $\theta t$:

    \begin{align}
        g=\frac{1}{n} \sum_{i=1}^n \nabla_\theta \mathcal{L}\left(x^{(i)}, y^{(i)}, \theta_t\right)
    \end{align}

    в то время как в \textbf{NAG} мы применяем скорость $vt$ к параметрам $\theta$ для вычисления
    промежуточных параметров $\theta$.
    Затем мы вычисляем градиент, используя промежуточные параметры

    \begin{equation}
        \begin{aligned}
            & \tilde{\theta}=\theta_t+\alpha v_t \\
            & g_{N A G} =\frac{1}{n} \sum_{i=1}^n \nabla_\theta \mathcal{L}\left(x^{(i)}, y^{(i)}, \tilde{\theta}\right) \\
            & v_{t+1} = \alpha v_t - \eta g \\
            & \theta_{t+1} = \theta_t + v_{t+1}
        \end{aligned}\label{eq:equation}
    \end{equation}

    Мы можем рассматривать \textbf{NAG} (и в принципе параметр \textbf{accelerate}) как поправочный коэффициент
    для \textbf{Momentum}.

    Когда скорость обучения $\eta$ относительно велика,
    \textbf{NAG} допускают большую скорость затухания $\alpha$, чем \textbf{Momentum}, предотвращая при этом колебания.

\end{document}